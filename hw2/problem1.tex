%!TEX root = main.tex
\problem{1}

We used Kleene's algorithm \footnote{\url{wikipedia.org/wiki/Kleene's_algorithm}} as a reference for our solution.

\begin{enumerate}[(a)]

%TODO img

\item
\[ 
	R(i,j,0) = 
  \begin{cases} 
      a | e & \text{if } i = j \text{ and } \delta(Q_{i}, a) = Q_{i} \\
      a_{1} | a_{2} | ... | a_{m} & \text{if } i \neq j \text{ and } \delta(Q_{i}, a_{1}) = \delta(Q_{i}, a_{2}) = ...  = \delta(Q_{i}, a_{m}) = Q_{j}
   \end{cases}
\]

Where e is the empty string.
Because the automata is deterministic there can be only one transition from one state to itself in the case where $i = j$.

\item 

$R(i,j,k)	= R(i,j,k-1) | R(i, k, k - 1)R(k,k,k-1)^{*}R(k,i,k-1)$

For each triple of states $Q_{i}$, $Q_{j}$ and $Q_{k}$ the accepting regular expression is the concatenation between the regular expresion from $Q_{i}$ to $Q_{j}$ that goes through $Q_{i}$, $Q_{j}$ ... $Q_{k-1}$ but not $Q_{k}$ and the regular expression from $Q_{i}$ to $Q_{j}$ that goes through $Q_{1}$,$Q_{2}$ ... $Q_{k}$

The general case recurrence is similar to the Floyd - Warshall algorithm with the following differences:

\begin{itemize}
	\item instead of the minimum between the two choices (the path from $Q_{i}$ to $Q_{j}$ that goesn't go through $Q_{k}$ and the path that goes through $Q_{k}$), the algorithm concatenates the regular expressions from both, since we want to find all regexes that go from $Q_{0}$ to final states
	\item When computing the alternate path (that goes through $Q_{k}$), we also include $R(k,k,k-1)^{*}$ to account for kleene stars that go from $Q_{k}$ to $Q_{k}$
\end{itemize}

\item

\begin{algorithm}[H]
	\caption{polynomial-time algorithm to compute R(i, j, n) for all states i and j}
	\begin{algorithmic}
	  \Require \text{R be a $|Q| \times |Q|$ matrix}
	  \Require $(\sum, Q, q0, F, \delta)$
	  \Function{regular}{$R$, $i$, $j$}
	  	\For{$i \gets 1, |Q|$}
			\For{$j \gets 1, |Q|$}
				\State $R[i,j] \gets \oslash$ \Comment{empty language}
				\If{i = j}
					\State $R[i,j]= a | e \text{ where } \delta(Q_{i}, a) = Q_{i} $
				\Else
					 \State $R[i,j] = a_{1} | a_{2} | ... | a_{m}  \text{ where } \delta(Q_{i}, a_{1}) = \delta(Q_{i}, a_{2}) = ...  = \delta(Q_{i}, a_{m}) = Q_{j}$
				\EndIf
			\EndFor
		\EndFor
		
		\For{$k \gets 1, |Q|$}
			\For{$i \gets 1, |Q|$}
				\For{$j \gets 1, |Q|$}
					\State $R[i,j] \gets R[i,j] | R[i, k]R[k,k]^{*}R[k,i]$
				\EndFor
			\EndFor
		\EndFor
	  \EndFunction
	\end{algorithmic}
\end{algorithm}

Space: $O(|Q|^2)$. We can use an array instead of a matrix by not explicitly modelling k.
Thus, each element of the matrix represents R(i,j,k) during the $k$-th iteration.
After iterating through all the values of k the matrix represents R(i,j,n) where $n=|Q|$.

Runtime: $O(|Q|^3)$. This is due to the three nested loops. Each loop has $|Q|$ iterations.

\end{enumerate}