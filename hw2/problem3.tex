%!TEX root = main.tex
\problem{3}


\begin{enumerate}[(a)]

	\item In the worst case, every time the new selected element is the new minimum. \emph{In this case, RandomMin will execute line(*) every time it will run the loop and hence \textbf{line $(*)$ will be run $n$ times.} } 
	
	
	\item 
	
	Let’s assume that we have 3 numbers: $1,2,3$
	Let us try to find the probability of RandomMin entering into line * on 3rd iteration.
	Random selection can produce following perfmuations:
	
	\begin{itemize}
	
		\item 123
		\item 132
		\item 213
		\item 231
		\item 312
		\item 321
		
	\end{itemize}
	
	The only way RandomMin can execute into line $(*)$ is by choosing 231 and 321. 
	So we can fix the last element as the smallest element and permute the remaining two elemens to execute line (*). 
	We have 2! ways to do it. And total possibilities that can be chosen by RandomMin is 3! for 3 numbers. 
	So, the probability of entering into line (*) is  $\frac{2!}{3!} = \frac{1}{3} $
	
	Lets assume 4 numbers $1, 2, 3, 4$ and we are trying to find probability of executing line (*) on the 4th iteration. 
	We have 4! total permuations and in order to execute line(*) , we need to fix the last number as 1 (the smallest number) and permute remaining 3 numbers and this can be done in 3! different ways. 
	So, the probability of executing line (*) on 4th iteration is $\frac{3!}{4!} = \frac{1}{4}$

	\paragraph{Generalization}
	
	While doing the $k$-th iteration, we have total of $k$ numbers randomly chosen from $n$ total numbers. 
	The total possible ways to execute random min loop using k numbers is k!. 
	In order to execute line $(*)$ on the $k$-th iteration, the $k$-th number should be smallest and this can be done by RandomMin in (k-1)! different possible ways (keeping the  $k$-th value fixed with the smallest number and permuting  the remaining numbers).

	So, the probability of executing line $(*)$ in the $k$-th iteration is: $\frac{(k-1)!}{k!} = \frac{1}{k}$ \\
	\\
	\textbf{Hence the probability of executing line (*) during the $n$-th iteration is: $\frac{1}{n}$.} 
	
	\item
	
	We already know from (b) that the probability that line(*) is executed on the $k$-th iteration is 
	$Pr[\text{Line (*) executed on kth iteration}] = \frac{1}{k}$.
	
	The frequency of each element is 1.
	
	So, the expected number of executions is :
	
	$ E(x) = \sum\limits_{k=1}^n 1 * pr($ Line(*)$ $ is$ $ executed$ $ on$ $ kth$ $ iteration$)$
	
       $ = \sum\limits_{k=1}^n 1 * \frac{1}{k} $
       
       $ = \sum\limits_{k=1}^n \frac{1}{k} $
	
	\textbf{\emph{$ = 1 + \frac{1}{2} + ... \frac{1}{n} $}}
	
	\textbf{\emph{$ = Hn$, (nth harmonic number }}  \footnote{Harmonic Number, \url{http://en.wikipedia.org/wiki/Harmonic_number}}).

\end{enumerate}
