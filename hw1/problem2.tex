%!TEX root = hw1.tex

\problem{2}

For the simple, unburned problem, we are using algorithm~\ref{algo:pancakes-simple}.
The basic function of the algorithm is as follows.
In line 2, we are looking for the largest pancake in the stack (line~\ref{line:pancake:find}).
We then flip all the pancakes, starting with the largest one (line~\ref{line:pancake:flip-large}).
This puts that largest pancake on the top.
We then flip the whole stack, which puts the large pancake on the bottom (line~\ref{line:pancake:flip-all}).
We then do this recursively, for the remaining pancakes, that are on top of the largest one (line 4).

\begin{algorithm}
	\caption{The solution for the simple, unburned, problem}
	\begin{algorithmic}[1]
	  \Function{SortPancakes}{$P[1..n]$}
	  	\If{$n \leq 1$}
			\State \Return
		\EndIf
	  	\State $p \gets$ \Call{FindLargestPancake}{$P[1..n]$} \Comment{$O(n)$} \label{line:pancake:find}
		\State \Call{FlipPancakes}{$P[1..n]$,$p$} \Comment{$O(n)$} \label{line:pancake:flip-large}
		\State \Call{FlipPancakes}{$P[1..n]$,$0$} \Comment{$O(n)$} \label{line:pancake:flip-all}
		\State \Call{SortPancakes}{$P[2..n]$}
	  \EndFunction
	\end{algorithmic}
	\label{algo:pancakes-simple}
\end{algorithm}

\paragraph{Runtime analysis}
The function $SortPancakes$ is called once for every pancake in the stack.
This gives us a time complexity of $O(n)$.
Inside each call, we to find the largest pancake, which is of time complexity $O(n)$.
Therfore, the overall complexity is $O(n^{2})$.

\paragraph{Maximum number of flips}
For each recursive call, we do 2 flips.
For the last pancake, we don't do any flips, since it's already the smallest and in place.
Therefore, we have a total of $2n-2$ flips.

For the case when we have to deal with burned pancakes, we are modifying algorithm~\ref{algo:pancakes-simple} by adding on more conditional at line~\ref{line:pancake:check-burned}.
The results is algorithm~\ref{algo:pancakes-burned}.

\begin{algorithm}
	\caption{The solution for the simple, unburned, problem}
	\begin{algorithmic}[1]
	  \Function{SortPancakes}{$P[1..n]$} 
	  	\If{$n \leq 1$}
			\If{\Call{IsFacingUp}{$p$}} \Comment{$O(1)$} \label{}
				\State \Call{FlipPancakes}{$P[1..n]$,$p$} \label{line:pancake:flip-last-pancake}
			\EndIf
			\State \Return
		\EndIf
	  	\State $p \gets$ \Call{FindLargestPancake}{$P[1..n]$} \Comment{$O(n)$} \label{line:pancake:find}
		\State \Call{FlipPancakes}{$P[1..n]$,$p$} \Comment{$O(n)$} \label{line:pancake:flip-large}
		\If{\Call{IsFacingUp}{$p$}} \label{line:pancake:check-burned} \Comment{$O(1)$} \label{}
			\State \Call{FlipPancakes}{$P[1..m]$,$m$}  \label{line:pancake:flip-burned}
		\EndIf
		\State \Call{FlipPancakes}{$P[1..n]$,$0$} \Comment{$O(n)$} \label{line:pancake:flip-all}
		\State \Call{SortPancakes}{$P[2..n]$}
	  \EndFunction
	\end{algorithmic}
	\label{algo:pancakes-burned}
\end{algorithm}

\paragraph{Runtime analysis}
The introduced conditional on line~\ref{line:pancake:check-burned} can be executed in constant time.
Therefore, the time complexity of the algorithm is unchanged at $O(n^2)$.

\paragraph{Maximum number of flips}
To guarantee the orientation, we might need to flip the pancake one extra time, as shown in line~\ref{line:pancake:flip-burned}.
Therefore, the maximum number of flips becomes $3n-2$.
Notice that while the last pancake is guaranteed to be in the correct position, it's orientation might be incorrect.
Hence, we might need to do one flip to orient it the correct way, as shown in line~\ref{line:pancake:flip-last-pancake}.